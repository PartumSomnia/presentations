%% --------------------------------------------------------------
%%
%% I N T R O D U C T I O N
%%
%% --------------------------------------------------------------

\begin{frame}{Introduction}  %% ---------- Intro/motivation 
    
    \begin{tikzpicture}[overlay,remember picture]
        
        \uncover<1->{ % <-> |
            \node (t1) [anchor=center,scale=1,opacity=1] at ([shift={(-0.0cm,1.2cm)}]current page.center){
                \parbox{1.1\textwidth}{
                    \begin{itemize}
                        \item Electrons, moving in a magnetic field emit "curavature" radiation. 
                        \item Relativistic (non-relativistic) electrons emit synchrotron (cyclotron)
                        \item Both limits can be treated analytically. Trnasion -- cannot. 
                        \item Cyclotron emission is common in solar flare modelling. 
%                        \item If magnetic fields are very strong, cyclotron harmonics develope fine structure. 
                        \item (i) Neglecting quantum effects. 
                        \item Cyclotron radiation: observer sees specturm composed of delta factions at $\omega_0$; % orbital frequency.
                        \item As electron speeds up: observer sees additional Fourier components (harmonics), $\omega_0 \propto \gamma_e^{-1}$, emission becomes more beamed in the direction of motion. 
                        \item For a realtivistc electron: spectrim is a close series of delta functions (continous) 
                        up to $\omega \sim \omega_{b}\gamma_e^2$ with universal shape. % cyclotron frequency $omega_b = eB/m_ec$
                    \end{itemize}
            }};
        }
        
%        
%        \uncover<1->{ % <-> |
%            \node (t1) [anchor=center,scale=1,opacity=1] at ([shift={(0.0cm,-3.8cm)}]current page.center){
%                \parbox{1.1\textwidth}{
%                    \red{Questions}: remnant's lifetime; ejecta properties; \rproc{} cites
%            }};
%        }
        
    \end{tikzpicture}
\end{frame}

% =============================================================================================

\begin{frame}{Introduction}  %% ---------- Intro/motivation 
    
    \begin{tikzpicture}[overlay,remember picture]
        
        
        \uncover<1->{ % <-> |
            \node (t1) [anchor=center,scale=1,opacity=1] at ([shift={(-0.0cm,-0.2cm)}]current page.center){
                \parbox{1.1\textwidth}{
                    % power = energy/time/steradian/frequency
                    \begin{equation*}
                        \boldsymbol{\eta}(\boldsymbol{\beta},\theta)d\omega = \frac{e^2 \omega^2}{2 \pi c}\Bigg[ \sum_{m=1}^{\infty} \Bigg( \frac{\cos\theta - \beta_{||}}{\sin \theta} \Bigg)^2 J_m^2(x) + \beta_{\perp}^2 J_m^{'2}(x) \Bigg]\delta(y) d\omega
                    \end{equation*}
                    where $x = (\omega/\omega_0)\beta_{\perp}\sin\theta$
                    $y=m\omega_0 - \omega(1 - \beta_{||}\cos\theta)$,
                    $\delta(y)$ is the delta function $J_m(x)$ is the Bessel function, 
                    $\beta_{||} = \beta\cos\theta_b$, $\beta_{\perp}=\beta\sin\theta_b$, 
                    with $\theta_b=\angle(\boldsymbol{\beta},\boldsymbol{B})$.
                    \begin{itemize}
                        \item Cyclotron, $m\beta \ll 1$: Expand $J(x)$ to lowest $x$, for successive harminics. 
                        \item Synchrotron $\gamma_e\ gg 1$ : $J(x)$ approximated by modified Bessel function
                    \end{itemize}
                    \begin{equation*}
                        \frac{dE}{d\omega} = \frac{\sqrt{3}e^3 B \sin\theta_p}{2 \pi m_e c^2}
                        \frac{\omega}{\omega_c}\int_{\omega/\omega_c}^{\infty}F_{5/3}(\xi)d\xi,
                    \end{equation*}
                    where $\omega_c = (3/2) \gamma^2 \omega_b \sin \theta_p$
                    \begin{equation*}
                        L_{\omega}=\frac{dE}{d\omega}=2\pi\int_0^1d\beta n(\beta) \int_0^1 d(\cos\theta_p) \int_{-1}^{1}d(\cos\theta)\boldsymbol{\eta}_{\boldsymbol{\beta},\theta}.
                    \end{equation*}
            }};
        }
        %        
        %        \uncover<1->{ % <-> |
            %            \node (t1) [anchor=center,scale=1,opacity=1] at ([shift={(0.0cm,-3.8cm)}]current page.center){
                %                \parbox{1.1\textwidth}{
                    %                    \red{Questions}: remnant's lifetime; ejecta properties; \rproc{} cites
                    %            }};
            %        }
        
    \end{tikzpicture}
\end{frame}