%% --------------------------------------------------------------------------------------------
%%
%%
%%
%% --------------------------------------------------------------------------------------------
\section{Disk-Remnant}
\begin{frame}{Angular momentum flux\footcite{Nedora:2019jhl}} %% ---------- title 
\begin{tikzpicture}[overlay,remember picture]
\uncover<1->{ % <-> |
        \node (t1) [anchor=center,scale=1,opacity=1] at ([shift={(-3.2cm,1.2cm)}]current page.center){
                \parbox{0.65\textwidth}{
                        The angular momentum flux in the cylindrical coordinates.% $x^i=(r,\phi,z)$.
                        % $\phi^{\nu}$ is the generator of the rotations in the orbital plane.
                        \begin{equation*}
                            \alpha\sqrt{\gamma}T^r _{\nu}\phi^{\nu} =
                            \alpha\sqrt{\gamma}\rho h W^2 (v^{r}v_{\phi})\, ,
                            \end{equation*}
                        assiming $\phi^{\mu} = (\partial_{\phi})^{\mu}$ is a Killing vector. % angular momentum is conserved
                }};
    }

\uncover<1-1>{ % <-> |
        \node (img1) [anchor=center,scale=1,opacity=1] at ([shift={(5.3cm,-0.6cm)}]current page.center){
                \parbox{0.5\textwidth}{
                        \includegraphics[height=6cm]{phd_figs/raycasting_smooth_cropped.pdf}
                        
                        \hspace*{2.8mm}Spiral arms within the disk
                }};
    }

%shocks are generated at the collisional interface of the \acp{NS} cores,
%as well as, tidal torques exerted by the non axisymmetric remnant result in a formation
%of the disk.
%Matter, energy and angular momentum are injected into the disk as the spiral 
%density waves propagate outwards from the mass-shedding \ac{MNS} remnant
\uncover<1->{ % <-> |
        \node (t2) [anchor=center,scale=1,opacity=1] at ([shift={(-3.5cm,-1.8cm)}]current page.center){
                \parbox{0.6\textwidth}{
                        \begin{itemize}
                                \item Shocks and tidal torques exerted from non-axisymmetric remnant $\rightarrow$ disk;
                                \item spiral arms is a generic hydrodynamic effect.
                            \end{itemize}
                        Injection of matter, energy and angular momentum.
                }};
    }

\end{tikzpicture}

\end{frame}
%
%
%%% ----------------------------------------------------------------
%%%
%%% Angular momentum trasport
%%%
%%% ----------------------------------------------------------------
%
\begin{frame}{Angular momentum transport in \pmerg{} remnant\footcite{Nedora:2020pak}}%$^\text{\textcolor{gray}{\citep{Nedora:2020pak}}}$} %% ---------- title 

\begin{tikzpicture}[overlay,remember picture]

\uncover<1->{ % <-> |
        \node (img2) [anchor=center,scale=1,opacity=1] at ([shift={(4.2cm,-0.4cm)}]current page.center){
                \parbox{0.5\textwidth}{
                        \includegraphics[height=6.0cm]{phd_figs/remnant/evol_jflux_2d_DD2_M13641364_M0_SR_R1.pdf}
                        
                        \hspace*{2.8mm}Ang. mom. transport through disk
                }};
    }
%shocks are generated at the collisional interface of the \acp{NS} cores,
%as well as, tidal torques exerted by the non axisymmetric remnant result in a formation
%of the disk.
%Matter, energy and angular momentum are injected into the disk as the spiral 
%density waves propagate outwards from the mass-shedding \ac{MNS} remnant

\uncover<1->{ % <-> |
        \node (t2) [anchor=center,scale=1,opacity=1] at ([shift={(-3.8cm,-0.8cm)}]current page.center){
                \parbox{0.6\textwidth}{
                        \begin{itemize}
                                \item The $J$ is transported via spiral waves induced by $m=\{1,2\}$ modes.
                                \item Spiral density modes inject ${\sim}0.1-0.4\,\Msun$ into the disk
                                \item ${\sim}50\%$ of $J$ remnant $\rightarrow$ disk
                                \item Mass injection is stronger in models with stiffer EOS. 
                                \item Accretion, mass-shedding, weak processes
                                % \item Larger temperatures lead to lower rotational frequency at which the mass 
                                % shedding occurs% \citep{Kaplan:2013wra}. 
                            \end{itemize}
                }};
    }

\end{tikzpicture}

\end{frame}











%% ----------------------------------------------------------------
%%
%% Long-term Evolution
%%
%% ----------------------------------------------------------------



\begin{frame}{Origin on the spiral-wave wind\footcite{Nedora:2019jhl,Nedora:2020pak}}%{Long term evolution \& remnant's fate}%$^\text{\textcolor{gray}{\citep{Nedora:2020pak}}}$} %% ---------- title 

%As the disk expands and cools, the recombination of nucleons into alpha particles 
%starts to take place. The energy, released in recombination, might be sufficient 
%for the outermost layers to become unbound, generating an outflow and contributing 
%to the disk depletion \citep{Beloborodov:2008nx,Lee:2009uc,Fernandez:2013tya}.
%%
%This process however is expected to take place on timescales, longer than
%those that are simulated here. On the $\sim100$~ms timescale, however, the outflows 
%are driven by the neutrino heating, above the remnant, the so-called neutrino-driven 
%wind (\nwind; \citep{Dessart:2008zd,Perego:2014fma,Just:2014fka}), and by the dynamical 
%interactions between the \ac{MNS} remnant and the disk, the \swind{} \citep{Nedora:2019jhl}.
%%
%We discuss the properties of the \nwind{} found in our simulations
%in Sec.~\ref{sec:bns_sims:nwind} and the properties of the 
%\swind{} in Sec.~\ref{sec:bns_sims:sww}.

\begin{tikzpicture}[overlay,remember picture]
    
    %We evaluate the amount of angular momentum lost to \acp{GW} following the 
    %\citet{Damour:2011fu,Bernuzzi:2012ci,Bernuzzi:2015rla}\footnote{
        %    The radiated angular momentum is computed from the 
        %    multipole loments $N_{lm}$ for the \ac{NR} complex "news function" at infinity. 
        %    The $J_{z;\text{rad}}(t)$ is computed as \citep{Damour:2011fu} 
        %    \begin{equation*}
            %    \Delta J_{z\text{rad}}(t) \frac{1}{16}\sum_{l,m}^{l_{max}}\int_{t_0}^{t} dt' m \mathcal{L}[h_{lm}(t')(N_{lm}(t'))^*],
            %    \end{equation*}
        %    where $h_{lm}$ is the multipolar metric waveform, 
        %    $N_{lm}(t) = dh_{lm}(t) / dt$, the news function, and $l_{max}=8$.
        %    The $J$ loss is metric dependent ($h$).
        %    The $h$ (strain???) is computed from $\Psi_4(t) = dN/dt = d^2h/dt^2$ by a 
        %    frequency-domain integration procedure with a low-frequency cut 
        %    $\omega_0 = 0.032/(m_1+m_2)$.
        %    The routines used for the calculation are taken from the scientific library
        %    \texttt{scidata}, available at \url{https://bitbucket.org/dradice/scidata}.
        %}.
    
    \uncover<1->{ % <-> |
        \node (t1) [anchor=center,scale=1,opacity=1] at ([shift={(-3.4cm,1.2cm)}]current page.center){
            \parbox{0.64\textwidth}{
                With Fourier mode decomposition, $e^{-i m\phi}$, of the Eulerian rest-mass density \textbf{we find:}
                %Modes analysis for exemplary equal-mass long-live and short-lived
                %                    %        remnants. The evolution of the $m=2$ and the $m=1$ monitored by
                %                    %        Eq.~\eqref{eq:modes} is shown for the DD2 and LS220 remnant with and
                %                    %        without turbulent viscosity. The $m=2$ mode in the long-lived
                %                    %        remnant is strongly damped by the emission of gravitational
                %                    %        radiation and becomes comparable to the $m=1$ mode on a timescale of
                %                    %        ${\gtrsim}20\,$ms. Turbulent viscosity sustain the $m=2$ mode for
                %                    %        a longer period. The $m=2$ mode is instead dominant to collapse in
                %                    %        the short-lived remnant.
                %                    %        (Adopted from \cite{Nedora:2020pak})
                \begin{itemize}
                    %\item Remnant is \textbf{not axisymmetric}, (prominant $m=2$ and $m=1$ modes); 
                    \item Prominent $m=2$ and $m=1$ modes; 
                    \item One-armed, $m=1$, persist till ${\sim}100$~ms;
                    \item Injection of matter, energy, angular momentum, $J$;
                    %                        \item Bar-shaped $m=2$ dominats the \textbf{early} phase
                    %                        and later $m=2$ is damped by the emission of \acp{GW}
                    %                        \item One-armed $m=1$ is present on ${\sim}100$~ms 
                    %                        \item $m=1$ mode is stronger if \mr{} is higher
                    %                        \item And as the $m=1$ modes are not efficiently damped$^\text{\textcolor{gray}{\citep{Paschalidis:2015mla,Radice:2016gym,Lehner:2016wjg,East:2016zvv}}}$
                    %%    \item Not axisymmetric. Angular momentum redistribution, \acp{GW} losses %\cite{Bernuzzi:2015opx,Radice:2018xqa}
                    %%    \item Long-term evolution driven by viscous processes \& weak interactions
                    %%    \item After \ac{GW} losses \ac{MNS} have excess in $M$ with respect to the cold, rigid.rot equilibria %\citep{Radice:2018xqa}.
                    %%    \item \ac{MNS} remnant evolves towards stability at the mass-shedding limit
                    %%    \item Collapse to a \ac{BH} depends on post-merger state, temperature and composition
                    %\begin{itemize}
                    %                \item Shocks and tidal torques exerted from non-axisymmetric remnant $\rightarrow$ disk;
                    %                \item spiral arms is a generic hydrodynamic effect.
                    %            \end{itemize}
                %            Injection of matter, energy and angular momentum.
            \end{itemize}
    }};
}
\uncover<1->{ % <-> |
    \node (t1) [anchor=center,scale=1,opacity=1] at ([shift={(-3.4cm,-1.9cm)}]current page.center){
        \parbox{0.65\textwidth}{
            \begin{itemize}
                % Viscosity foces rotating equilibria to rotate rigidly
                % Non-axial symmetry -- gravitational radiation
                % ADM mass and angular momentum must be conserved 
                % J_GW is computed from the Weil scalar (Psi4) decomposed into s=-2 spin-weighted spherical harmonics
                % Angular momentum, trasported away by GWs
                % Differential rotation is reduced by the non-axisymmetric torques and GWs losses that redistribute angular momentum from inner to the outer regions
                \item Excess in baryonic mass, $M_b$, and $J$;
                % \item \acp{GW} remove some $J$
                \item $J$ is lost to \acp{GW} and outflows
                %\item Massive outflows from expanded disk
                %driven by $m=\{1,2\}$ and $J$ transport
                \item \textbf{Spiral-wave wind}, driven by $m=\{1,2\}$ instabilities, removes $J$ and $M_b$
                \item The remnant can reach \textbf{stable configuration}
                %\item Study of remnant's fate requires long 3D \ac{NR} simulations
            \end{itemize}
    }};
}

\uncover<1->{ % <-> |
    \node (img1) [anchor=center,scale=1,opacity=1] at ([shift={(4.6cm,-0.4cm)}]current page.center){
        \parbox{0.5\textwidth}{
            \includegraphics[height=6cm]{phd_figs/remnant/dens_modes/modes_rho_dd2.pdf}\\
            \hspace*{10mm}Demsity modes%\footcite{Nedora:2020pak}
            %            Baryon mass vs angular momentum diagram for the BLh $q=1$ remnant.
            %            The colored diamond marks the baryonic mass and angular momentum at the end
            %            of the dynamical gravitational-wave dominated phase.
            %            After the GW phase, the evolution is driven by the massive outflows.
            %            The solid black line is the $M_b$ and $J$ estimated from the 3D data
            %            integrals under the assumption of axisymmetry.
            %            The green dashed line is a conservative estimate
            %            of the mass ejection and a possible trajectory for the viscous
            %            evolution as estimated in \citet{Radice:2018xqa}. The crosses are
            %            a linear extrapolation in time of the solid black line. The gray
            %            shaded region is the region of stability of rigidly rotating NS equilibria.
            %            Adopted from \cite{Nedora:2020pak}
    }};
}

%\uncover<1->{ % <-> |
    %    \node (t2) [anchor=center,scale=1,opacity=1] at ([shift={(-3.8cm,-1.8cm)}]current page.center){
        %        \parbox{0.6\textwidth}{
            %            \begin{itemize}
                %            \item The long-term evolution of the disk is driven by its interaction with the \ac{MNS} 
                %            remnant and cooling.
                %            \item Accretion vs. mass shedding
                %            \item spiral density waves inject mass and energy into disk and it heats up \& expands
                %            \item Softness \ac{EOS} (strong desntiy modes) \& high temps (allowed by \ac{EOS}) $\rightarrow$ mass-shedding
                %            \end{itemize}
            %    }};
    %}

\end{tikzpicture}

\end{frame}


%\section{Disk}
\begin{frame}{Disk mass evolution} %% ---------- title 

%During the merger, shocks, generated at the collisional interface of the \ac{NS} cores,
%as well as, tidal torques exerted by the non axisymmetric remnant, result in a formation
%of the disk. Matter, energy and angular momentum are injected into the disk as the spiral 
%density waves propagate outwards from the mass-shedding \ac{MNS} remnant 
%\citep{Bernuzzi:2015opx,Radice:2018xqa}.

\begin{tikzpicture}[overlay,remember picture]

\uncover<1->{ % <-> |
    \node (t1) [anchor=center,scale=1,opacity=1] at ([shift={(-3.6cm,1.0cm)}]current page.center){
        \parbox{0.6\textwidth}{
            The disk baryonic mass is computed as the volume integral of the conserved rest-mass density 
            %
            \begin{equation*}
            \label{eq:method:mdisk}
            M_{\text{disk}} = \int_{\rho>10^{13}\,\text{\gcm}} \sqrt{\gamma}~W\rho \, \dd^3 x
            \end{equation*}
    }};
}

\uncover<1->{ % <-> |
    \node (img1) [anchor=center,scale=1,opacity=1] at ([shift={(4.8cm,-0.4cm)}]current page.center){
        \parbox{0.6\textwidth}{
            \includegraphics[height=6.5cm]{phd_figs/disk/total_disk_mass_evo.pdf}
    }};
}

%\uncover<1->{ % <-> |
%    \node (t2) [anchor=center,scale=1,opacity=1] at ([shift={(-3.8cm,-2.2cm)}]current page.center){
%        \parbox{0.6\textwidth}{
%            \begin{itemize}
%            \item The long-term evolution of the disk is driven by its interaction with the remnant and cooling.
%            \item Accretion vs. mass shedding
%            \item spiral density waves inject mass and energy into disk and it heats up \& expands
%            \item Soft EOS (strong desntiy modes) \& high temps (allowed by EOS) $\rightarrow$ mass-shedding
%            \end{itemize}
%    }};
%}

\end{tikzpicture}

\end{frame}

%% --------------------------------------------------------------------------------------------
%%
%%
%%
%% --------------------------------------------------------------------------------------------

\begin{frame}{The evolution of the mass-weighted electron fraction} %% ---------- title 

\begin{tikzpicture}[overlay,remember picture]

\uncover<1->{ % <-> |
    \node (t1) [anchor=center,scale=1,opacity=1] at ([shift={(-0.8cm,1.6cm)}]current page.center){
        \parbox{1.0\textwidth}{
            \begin{itemize}
            \item During the formation, shocks and spiral waves raise the disk $Y_e\sim0.25$
            \item The bulk, shielded from neutrinos, returns to $Y_e\lesssim0.1$.
            \item The outer part of the disk is subjected to the strong irradiation and reaches $Y_e\sim0.4$.
            \end{itemize}
    }};
}

\uncover<1->{ % <-> |
    \node (img1) [anchor=center,scale=1,opacity=1] at ([shift={(4.5cm,-1.7cm)}]current page.center){
        \parbox{0.4\textwidth}{
            \includegraphics[height=5.2cm]{phd_figs/disk/final_disk_timecorr_blh_q1_Lk.pdf}
    }};
}

\uncover<1->{ % <-> |
    \node (img1) [anchor=center,scale=1,opacity=1] at ([shift={(-2cm,-1.7cm)}]current page.center){
        \parbox{0.4\textwidth}{
            \includegraphics[height=5.2cm]{phd_figs/disk/final_disk_timecorr_ls220_q14_LK.pdf}
    }};
}

\end{tikzpicture}

\end{frame}

%\includegraphics[width=0.49\textwidth]{disk/final_disk_timecorr_blh_q1_Lk.pdf}
%\includegraphics[width=0.49\textwidth]{disk/final_disk_timecorr_ls220_q14_LK.pdf}

%% --------------------------------------------------------------------------------------------
%%
%%
%%
%% --------------------------------------------------------------------------------------------


\begin{frame}{Final disk structure} %% ---------- title 

\begin{tikzpicture}[overlay,remember picture]

\uncover<1->{ % <-> |
    \node (t1) [anchor=center,scale=1,opacity=1] at ([shift={(-0.8cm,1.8cm)}]current page.center){
        \parbox{1.0\textwidth}{
            \begin{itemize}
            \item Low $Y_e$ inner disk
            \item High $Y_e$ outer disk
            \end{itemize}
    }};
}

\uncover<1->{ % <-> |
    \node (img1) [anchor=center,scale=1,opacity=1] at ([shift={(3.5cm,-1.5cm)}]current page.center){
        \parbox{0.5\textwidth}{
            \includegraphics[height=5.4cm]{phd_figs/disk/final_structure/slice_xz_entr_ye_blh_q1_rl3.pdf}
    }};
}

\uncover<1->{ % <-> |
    \node (img1) [anchor=center,scale=1,opacity=1] at ([shift={(-3.5cm,-1.5cm)}]current page.center){
        \parbox{0.5\textwidth}{
            \includegraphics[height=5.4cm]{phd_figs/disk/final_structure/slice_xy_entr_ye_blh_q1_rl3.pdf}
    }};
}

\end{tikzpicture}

\end{frame}

%\includegraphics[width=0.49\textwidth]{disk/final_structure/slice_xz_entr_ye_blh_q1_rl3.pdf}
%\includegraphics[width=0.49\textwidth]{disk/final_structure/slice_xy_entr_ye_blh_q1_rl3.pdf}

%% --------------------------------------------------------------------------------------------
%%
%%
%%
%% --------------------------------------------------------------------------------------------


\begin{frame}{Final disk properties \& composition} %% ---------- title 

\begin{tikzpicture}[overlay,remember picture]

\uncover<1->{ % <-> |
    \node (t1) [anchor=center,scale=1,opacity=1] at ([shift={(-0.8cm,1.8cm)}]current page.center){
        \parbox{1.0\textwidth}{
            \begin{itemize}
            \item Bimodal structure in $\langle Y_e \rangle$ and $\langle s \rangle$. Second peak \ac{EOS} \& \mr{} dependent
            \item High $Y_e$ outer disk
            \end{itemize}
    }};
}

\uncover<1->{ % <-> |
    \node (img1) [anchor=center,scale=1,opacity=1] at ([shift={(-0.7cm,-1.5cm)}]current page.center){
        \parbox{1.0\textwidth}{
            \includegraphics[height=5.2cm]{phd_figs/disk/final_structure/disk_hist_shared.pdf}
    }};
}

%\uncover<1->{ % <-> |
%    \node (img1) [anchor=center,scale=1,opacity=1] at ([shift={(-4.0cm,-1.5cm)}]current page.center){
%        \parbox{0.5\textwidth}{
%            \includegraphics[height=5cm]{disk/final_structure/slice_xy_entr_ye_blh_q1_rl3.pdf}
%    }};
%}

\end{tikzpicture}

\end{frame}



% ---------------------------------------------------------
%
% Spiral wave wind
%
% --------------------------------------------------------
\section{Winds}
\begin{frame}{Spiral-wave wind -- Mass}
\begin{tikzpicture}[overlay,remember picture]
\uncover<1->{ % <-> |
    \node (t1) [anchor=center,scale=1,opacity=1] at ([shift={(-3.0cm,0.5cm)}]current page.center){
        \parbox{0.7\textwidth}{
            %Bernoulli criterion ($hu_t < -1$, stationary flow)\\
            \begin{itemize}
            \item is an indicator of the remnant lifetime,
            \item is larger if \ac{EOS} is softer (as $m=1$ is stronger) % stiffer \ac{EOS} leads to a more massive disk 
            \item is larger if $q>1$
            \item increases if turbulent viscocity is present
            \end{itemize}
            \textbf{$M_{\text{wind}}$ is larger for more extended disks}. % that in turn depend on thermal pressure
    }};
}
\uncover<1->{ % <-> |
    \node (img1) [anchor=center,scale=1,opacity=1] at ([shift={(4.2cm,-0.4cm)}]current page.center){
        \parbox{0.5\textwidth}{
            \includegraphics[height=6.5cm]{phd_figs/ejecta_postdyn/wind_mass_flux.pdf}
    }};
}
\end{tikzpicture}
\end{frame}

\begin{frame}{Spiral-wave wind \& neutrino-driven wind} %% ---------- title 

%% --------------------------------------------------------------------------------------------
%%
%%
%%
%% --------------------------------------------------------------------------------------------

%    \includegraphics[width=0.49\textwidth]{slices/slice_xz_ye_hu_1.pdf}
%\includegraphics[width=0.49\textwidth]{slices/slice_xz_abs_energy_hu_3.pdf}
%\includegraphics[width=0.49\textwidth]{slices/slice_xy_ye_hu_1.pdf}
%\includegraphics[width=0.49\textwidth]{slices/slice_xy_abs_energy_hu_3.pdf}

\begin{tikzpicture}[overlay,remember picture]

\uncover<1->{ % <-> |
    \node (t1) [anchor=center,scale=1,opacity=1] at ([shift={(-0.8cm,1.8cm)}]current page.center){
        \parbox{1.0\textwidth}{
            \begin{itemize}
            %\item Absorption of electron neutrinos raises $Y_e$ of the polar outflow, drives outflow.
            \item Neutrino-wind is not a steady state outflow, ejecting ${\sim}10^{-3}-10^{-4}M_{\odot}$.
            %% The strongest neutrino heating occurs in the vicinity of the remnant at densities $\rho\sim10^{11}$~\gcm,
            \end{itemize}
    }};
}

\uncover<1->{ % <-> |
    \node (img1) [anchor=center,scale=1,opacity=1] at ([shift={(3.5cm,-1.5cm)}]current page.center){
        \parbox{0.5\textwidth}{
            \includegraphics[height=5.5cm]{phd_figs/slices/slice_xz_ye_hu_1.pdf}
    }};
}
%\uncover<2->{ % <-> |
%    \node (img1) [anchor=center,scale=1,opacity=1] at ([shift={(4.0cm,-1.5cm)}]current page.center){
%        \parbox{0.5\textwidth}{
%            \includegraphics[height=5cm]{slices/slice_xy_ye_hu_1.pdf}
%    }};
%}

%%% Right: $-hu_0$ and the absorption energy rate $Q_{\text{abs};\:\bar{\nu}_e}$ 
%%% of electron antineutrinos normalized to the fluid density $D$.
%%%  heating energy rate due to electron anti-neutrino absorption $Q_{\text{abs};\:\bar{\nu}_e}$
\uncover<1->{ % <-> |
    \node (img1) [anchor=center,scale=1,opacity=1] at ([shift={(-3.5cm,-1.5cm)}]current page.center){
        \parbox{0.5\textwidth}{
            \includegraphics[height=5.5cm]{phd_figs/slices/slice_xz_abs_energy_hu_3.pdf}
    }};
}
%\uncover<2->{ % <-> |
%    \node (img1) [anchor=center,scale=1,opacity=1] at ([shift={(-4.0cm,-1.5cm)}]current page.center){
%        \parbox{0.5\textwidth}{
%            \includegraphics[height=5cm]{slices/slice_xy_abs_energy_hu_3.pdf}
%    }};
%}

\end{tikzpicture}

\end{frame}