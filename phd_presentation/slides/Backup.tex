%% --------------------------------------------------------------------------------------------
%%
%%
%%
%% --------------------------------------------------------------------------------------------

\begin{frame}{Disk mass evolution} %% ---------- title 

%During the merger, shocks, generated at the collisional interface of the \ac{NS} cores,
%as well as, tidal torques exerted by the non axisymmetric remnant, result in a formation
%of the disk. Matter, energy and angular momentum are injected into the disk as the spiral 
%density waves propagate outwards from the mass-shedding \ac{MNS} remnant 
%\citep{Bernuzzi:2015opx,Radice:2018xqa}.

\begin{tikzpicture}[overlay,remember picture]

\uncover<1->{ % <-> |
    \node (t1) [anchor=center,scale=1,opacity=1] at ([shift={(-3.8cm,1.0cm)}]current page.center){
        \parbox{0.6\textwidth}{
            The disk baryonic mass is computed as the volume integral of the conserved rest-mass density 
            %
            \begin{equation*}
            \label{eq:method:mdisk}
            M_{\text{disk}} = \int_{\rho>10^{13}\,\text{\gcm}} \sqrt{\gamma}~W\rho \dd^3 x
            \end{equation*}
    }};
}

\uncover<1->{ % <-> |
    \node (img1) [anchor=center,scale=1,opacity=1] at ([shift={(4.8cm,-0.2cm)}]current page.center){
        \parbox{0.5\textwidth}{
            \includegraphics[height=6cm]{phd_figs/disk/total_disk_mass_evo.pdf}
    }};
}

\uncover<1->{ % <-> |
    \node (t2) [anchor=center,scale=1,opacity=1] at ([shift={(-3.8cm,-2.2cm)}]current page.center){
        \parbox{0.6\textwidth}{
            \begin{itemize}
            \item The long-term evolution of the disk is driven by its interaction with the remnant and cooling.
            \item Accretion vs. mass shedding
            \item spiral density waves inject mass and energy into disk and it heats up \& expands
            \item Soft EOS (strong desntiy modes) \& high temps (allowed by EOS) $\rightarrow$ mass-shedding
            \end{itemize}
    }};
}

\end{tikzpicture}

\end{frame}

%% --------------------------------------------------------------------------------------------
%%
%%
%%
%% --------------------------------------------------------------------------------------------

\begin{frame}{The evolution of the mass-weighted electron fraction} %% ---------- title 

\begin{tikzpicture}[overlay,remember picture]

\uncover<1->{ % <-> |
    \node (t1) [anchor=center,scale=1,opacity=1] at ([shift={(-0.8cm,1.6cm)}]current page.center){
        \parbox{1.0\textwidth}{
            \begin{itemize}
            \item During the formation, shocks and spiral waves raise the disk $Y_e\sim0.25$
            \item The bulk, shielded from neutrinos, returns to $Y_e\lesssim0.1$.
            \item The outer part of the disk, is subjected to the strong irradiation and reaches $Y_e\sim0.4$.
            \end{itemize}
    }};
}

\uncover<1->{ % <-> |
    \node (img1) [anchor=center,scale=1,opacity=1] at ([shift={(5.5cm,-1.5cm)}]current page.center){
        \parbox{0.5\textwidth}{
            \includegraphics[height=5cm]{phd_figs/disk/final_disk_timecorr_blh_q1_Lk.pdf}
    }};
}

\uncover<1->{ % <-> |
    \node (img1) [anchor=center,scale=1,opacity=1] at ([shift={(-0.5cm,-1.5cm)}]current page.center){
        \parbox{0.5\textwidth}{
            \includegraphics[height=5cm]{phd_figs/disk/final_disk_timecorr_ls220_q14_LK.pdf}
    }};
}

\end{tikzpicture}

\end{frame}

%\includegraphics[width=0.49\textwidth]{disk/final_disk_timecorr_blh_q1_Lk.pdf}
%\includegraphics[width=0.49\textwidth]{disk/final_disk_timecorr_ls220_q14_LK.pdf}

%% --------------------------------------------------------------------------------------------
%%
%%
%%
%% --------------------------------------------------------------------------------------------


\begin{frame}{Final disk structure} %% ---------- title 

\begin{tikzpicture}[overlay,remember picture]

\uncover<1->{ % <-> |
    \node (t1) [anchor=center,scale=1,opacity=1] at ([shift={(-0.8cm,1.8cm)}]current page.center){
        \parbox{1.0\textwidth}{
            \begin{itemize}
            \item Low $Y_e$ inner disk
            \item High $Y_e$ outer disk
            \end{itemize}
    }};
}

\uncover<1->{ % <-> |
    \node (img1) [anchor=center,scale=1,opacity=1] at ([shift={(4.0cm,-1.5cm)}]current page.center){
        \parbox{0.5\textwidth}{
            \includegraphics[height=5cm]{phd_figs/disk/final_structure/slice_xz_entr_ye_blh_q1_rl3.pdf}
    }};
}

\uncover<1->{ % <-> |
    \node (img1) [anchor=center,scale=1,opacity=1] at ([shift={(-4.0cm,-1.5cm)}]current page.center){
        \parbox{0.5\textwidth}{
            \includegraphics[height=5cm]{phd_figs/disk/final_structure/slice_xy_entr_ye_blh_q1_rl3.pdf}
    }};
}

\end{tikzpicture}

\end{frame}

%\includegraphics[width=0.49\textwidth]{disk/final_structure/slice_xz_entr_ye_blh_q1_rl3.pdf}
%\includegraphics[width=0.49\textwidth]{disk/final_structure/slice_xy_entr_ye_blh_q1_rl3.pdf}

%% --------------------------------------------------------------------------------------------
%%
%%
%%
%% --------------------------------------------------------------------------------------------


\begin{frame}{Final disk properties \& composition} %% ---------- title 

\begin{tikzpicture}[overlay,remember picture]

\uncover<1->{ % <-> |
    \node (t1) [anchor=center,scale=1,opacity=1] at ([shift={(-0.8cm,1.8cm)}]current page.center){
        \parbox{1.0\textwidth}{
            \begin{itemize}
            \item Bimodal structure in $\langle Y_e \rangle$ and $\langle s \rangle$, second peak \ac{EOS} \& \mr{} dependent
            \item High $Y_e$ outer disk
            \end{itemize}
    }};
}

\uncover<1->{ % <-> |
    \node (img1) [anchor=center,scale=1,opacity=1] at ([shift={(0.0cm,-1.5cm)}]current page.center){
        \parbox{1.0\textwidth}{
            \includegraphics[height=5cm]{phd_figs/disk/final_structure/disk_hist_shared.pdf}
    }};
}

%\uncover<1->{ % <-> |
%    \node (img1) [anchor=center,scale=1,opacity=1] at ([shift={(-4.0cm,-1.5cm)}]current page.center){
%        \parbox{0.5\textwidth}{
%            \includegraphics[height=5cm]{disk/final_structure/slice_xy_entr_ye_blh_q1_rl3.pdf}
%    }};
%}

\end{tikzpicture}

\end{frame}



% ---------------------------------------------------------
%
% Spiral wave wind
%
% --------------------------------------------------------


\begin{frame}{Spiral-wave wind -- Mass}
\begin{tikzpicture}[overlay,remember picture]
\uncover<1->{ % <-> |
    \node (t1) [anchor=center,scale=1,opacity=1] at ([shift={(-3.1cm,0.5cm)}]current page.center){
        \parbox{0.7\textwidth}{
            Bernoulli criterion ($hu_t < -1$, stationary flow)\\
            \begin{itemize}
            \item is an idicator of the remnant lifetime,
            \item is larger if \ac{EOS} is softer (as $m=1$ is stronger) % stiffer \ac{EOS} leads to a more massive disk 
            \item is larger if $q>1$
            \item increases if turbulent viscocity is present
            \end{itemize}
            $M_{\text{wind}}$ is larger for more extended disks. % that in turn depend on thermal pressure
    }};
}
\uncover<1->{ % <-> |
    \node (img1) [anchor=center,scale=1,opacity=1] at ([shift={(4.2cm,-0.2cm)}]current page.center){
        \parbox{0.5\textwidth}{
            \includegraphics[height=6cm]{phd_figs/ejecta_postdyn/wind_mass_flux.pdf}
    }};
}
\end{tikzpicture}
\end{frame}

\begin{frame}{Spiral-wave wind \& neutrino-driven wind} %% ---------- title 

%% --------------------------------------------------------------------------------------------
%%
%%
%%
%% --------------------------------------------------------------------------------------------

%    \includegraphics[width=0.49\textwidth]{slices/slice_xz_ye_hu_1.pdf}
%\includegraphics[width=0.49\textwidth]{slices/slice_xz_abs_energy_hu_3.pdf}
%\includegraphics[width=0.49\textwidth]{slices/slice_xy_ye_hu_1.pdf}
%\includegraphics[width=0.49\textwidth]{slices/slice_xy_abs_energy_hu_3.pdf}

\begin{tikzpicture}[overlay,remember picture]

\uncover<1->{ % <-> |
    \node (t1) [anchor=center,scale=1,opacity=1] at ([shift={(-0.8cm,1.8cm)}]current page.center){
        \parbox{1.0\textwidth}{
            \begin{itemize}
            \item absorption of $\nu_e$ neutrinos raises the $Y_e$ of the polar outflow, drives outflow
            \item \nwind{} is not a steady state outflow, ${\sim}10^{-3}-10^{-4}M_{\odot}$
            %% The strongest neutrino heating occurs in the vicinity of the remnant at densities $\rho\sim10^{11}$~\gcm,
            \end{itemize}
    }};
}

\uncover<1->{ % <-> |
    \node (img1) [anchor=center,scale=1,opacity=1] at ([shift={(4.0cm,-1.5cm)}]current page.center){
        \parbox{0.5\textwidth}{
            \includegraphics[height=5cm]{phd_figs/slices/slice_xz_ye_hu_1.pdf}
    }};
}
%\uncover<2->{ % <-> |
%    \node (img1) [anchor=center,scale=1,opacity=1] at ([shift={(4.0cm,-1.5cm)}]current page.center){
%        \parbox{0.5\textwidth}{
%            \includegraphics[height=5cm]{slices/slice_xy_ye_hu_1.pdf}
%    }};
%}

%%% Right: $-hu_0$ and the absorption energy rate $Q_{\text{abs};\:\bar{\nu}_e}$ 
%%% of electron antineutrinos normalized to the fluid density $D$.
%%%  heating energy rate due to electron anti-neutrino absorption $Q_{\text{abs};\:\bar{\nu}_e}$
\uncover<1->{ % <-> |
    \node (img1) [anchor=center,scale=1,opacity=1] at ([shift={(-4.0cm,-1.5cm)}]current page.center){
        \parbox{0.5\textwidth}{
            \includegraphics[height=5cm]{phd_figs/slices/slice_xz_abs_energy_hu_3.pdf}
    }};
}
%\uncover<2->{ % <-> |
%    \node (img1) [anchor=center,scale=1,opacity=1] at ([shift={(-4.0cm,-1.5cm)}]current page.center){
%        \parbox{0.5\textwidth}{
%            \includegraphics[height=5cm]{slices/slice_xy_abs_energy_hu_3.pdf}
%    }};
%}

\end{tikzpicture}

\end{frame}