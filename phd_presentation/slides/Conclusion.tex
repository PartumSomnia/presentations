\begin{frame}{Conclusion}  %% ---------- Intro/motivation 

\begin{tikzpicture}[overlay,remember picture]

\uncover<1->{ % <-> |
    \node (t1) [anchor=center,scale=1,opacity=1] at ([shift={(-0.0cm,1.2cm)}]current page.center){
        \parbox{1.1\textwidth}{
            % Our analysis of NR BNS merger simulations showed
            \begin{itemize}
            \item a HMNS can become stable, releasing $M_b$, and $J$ in form of winds
            \item winds are fast and proton-rich -- for the blue kilonova
            %\item dynamical ejeca \& winds produce \rproc{} abundance pattern consistent with solar
            \item fast ejecta is a generic attribute (unless prompt collapse)
            \item kilonova afterglow from models with moderately soft EOS is compatible with obs.
            %\item statistically, ejecta parameters depend on physics input in simulations
            \end{itemize}
    }};
}

\uncover<2->{ % <-> |
    \node (t1) [anchor=center,scale=1,opacity=1] at ([shift={(-0.0cm,-2.5cm)}]current page.center){
        \parbox{1.1\textwidth}{
            Future work \& directions
            \begin{itemize}
            \item larger number of BNS models with advanced physics is required
            \item advanced models of EM counterparts (\eg, late kilonova)
            \item more types if EM counterparts (\eg, fall-back accretion, UV-precursor)
            \item preparation for large number of observations
            %\item adapting MM pipelines for new EM models, statistical method, machine learning
            \end{itemize}
            
    }};
}

\end{tikzpicture}
\end{frame}