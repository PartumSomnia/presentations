\begin{frame}{Conclusion}  %% ---------- Intro/motivation 

\begin{tikzpicture}[overlay,remember picture]

\uncover<1->{ % <-> |
    \node (t1) [anchor=center,scale=1,opacity=1] at ([shift={(-0.0cm,1.0cm)}]current page.center){
        \parbox{1.1\textwidth}{
            % Our analysis of NR BNS merger simulations showed
            \begin{itemize}
            \item A remnant born with an excess in baryonic mass and angular momentum can reach equilibrium configuration by shedding mass in a form of \textbf{\swind}. 
            \item The wind is less neutron-rich and contribute to the \textbf{"blue" kilonova}.
            %\item dynamical ejeca \& winds produce \rproc{} abundance pattern consistent with solar
            \item \textbf{Fast dynamical ejecta} originates at core bounces with properties dependent on \ac{EOS} and \mr{}.
            \item \textbf{Kilonova afterglow} in x-ray band is compatible with latest observations (weakly) constraining binary parameters to moderately soft \ac{EOS} and mass ratio ${\gtrsim}1$.
            %\item statistically, ejecta parameters depend on physics input in simulations
            \end{itemize}
            %sResult: 
    }};
}

\uncover<2->{ % <-> |
    \node (t1) [anchor=center,scale=1,opacity=1] at ([shift={(-0.0cm,-2.5cm)}]current page.center){
        \parbox{1.1\textwidth}{
            \textit{ Future work \& directions:}
            \begin{itemize}
            \item High resolution \ac{NR}-simulations with advanced physics for detailed ejecta investigation.
            \item Radiation transport - hydrodynamic kilonova models.
            \item \ac{GRB} and kilonova afterglows combined analysis.
            \item Other types of \ac{EM} counterparts.
            % \item preparation for large number of observations
            % \item adapting MM pipelines for new EM models, statistical method, machine learning
            \end{itemize}
            
    }};
}

\end{tikzpicture}
\end{frame}