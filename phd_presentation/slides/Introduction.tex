%% --------------------------------------------------------------
%%
%% I N T R O D U C T I O N
%%
%% --------------------------------------------------------------

\begin{frame}{Motivation}  %% ---------- Intro/motivation 

\begin{tikzpicture}[overlay,remember picture]

\uncover<1->{ % <-> |
    \node (t1) [anchor=center,scale=1,opacity=1] at ([shift={(-0.0cm,1.5cm)}]current page.center){
        \parbox{1.1\textwidth}{
            Binary neutron star (BNS) mergers are unique cosmic laboratories to study 
            \begin{itemize}
                \item theory of gravity in a strong field regime,
                \item properties of matter at supranuclear densities,
                \item cosmic chemical evolution,%\nuc{} of the heaviest elements in the Universe,
                %\item cosmology.%, using \ac{BNS} mergers as standard sirens
            \end{itemize}
    }};
}

\uncover<2->{ % <-> |
    \node (t1) [anchor=center,scale=1,opacity=1] at ([shift={(-0.0cm,-1.5cm)}]current page.center){
        \parbox{1.1\textwidth}{
            In August 2017, a BNS merger was observed as a source of 
        
            \begin{itemize}
                \item \acp{GW}, \GW{},
                \item thermal electromagnetic (EM) emission, kilonova (kN), \AT{},
                \item non-thermal \ac{EM} short gamma-ray burst (GRB), \GRB{}
            \end{itemize}
            
            for the first time. \\
            
%            Wealth of information about different physical processes occurring in BNS\\
            %\red{Questions}: remnant's lifetime; ejecta mass/composition; sources of \rproc{} elements
    }};
}

\uncover<3->{ % <-> |
    \node (t1) [anchor=center,scale=1,opacity=1] at ([shift={(0.0cm,-3.4cm)}]current page.center){
        \parbox{1.1\textwidth}{
%            \textcolor{black}{Major advanclemts} were maid in our understanding of related physical processes \\
            \textcolor{black}{ \textbf{Key questions}}: remnant fate; origin of the early kilonova emission; origin of the late \ac{GRB} afterglow; \\
            \textbf{What is NS equation of state (\ac{EOS})?}
    }};
}


\end{tikzpicture}
\end{frame}
