\begin{frame}{}  %% ---------- Intro/motivation 

\centering \Huge{Thank you for your attention}

\end{frame}







%% =======================================================
%%
%%                   Method
%%
%% =======================================================

\begin{frame}{Method} %% ---------- title 

\begin{tikzpicture}[overlay,remember picture]

\uncover<1->{ % <-> |
    \node (t1) [anchor=center,scale=1,opacity=1] at ([shift={(-3.9cm,-0.7cm)}]current page.center){
        \parbox{0.53\textwidth}{
            Passing through the cold ISM, shock front 
            \begin{itemize}
            \item randomizes the particles' $\vec{\upsilon}$,
            \item compresses the plasma,
            \item amplifies the magnetic fields 
            \item accelerates the inbound particles. % to a power-law distribution function.
            \end{itemize}
            
            \textbf{Dynamics} is guvenred by 
            %A relativistic shock propagating into a cold upstream medium.
            $3$ conservation laws: baryon number, energy, momentum fluxes across shock front.
            %            The latter two $T^{\mu\nu} = (\rho' c^2 + p') u^{\mu} u^{\nu} + p' g^{\mu\nu},$
            %
            %            These equations can be written as 
            %
            %            \begin{equation} % subequations
            %            \begin{aligned} % align
            %            \frac{e_2'}{n_2'} &= (\gamma_{21} - 1)m_p c^2 \\
            %            \frac{n_2'}{n_1'} &= \frac{\hat{\gamma}\gamma_{21} + 1}{\hat{\gamma}-1} \\
            %            \gamma_{1s}^2 &= \frac{(\gamma_{21} + 1) [\hat{\gamma}(\gamma_{21}-1)+1]^2}{\hat{\gamma}(2-\hat{\gamma})(\gamma_{21}-1)+2}
            %            \end{aligned} % align
            %            \label{eq:afterglow:blast}
            %            \end{equation} % subequations
            %
            %            \begin{figure*}[t]
            %            \centering 
            %            \includegraphics[width=0.45\textwidth]{Fig_8_KZ.pdf}
            %            \caption{
            %                This is a schematic sketch of a pair of shocks produced when a relativistic
            %                jet from a \ac{GRB} collides with the \ac{CBM}, as viewed from the
            %                rest frame of unshocked \ac{CBM}. Regions 2 \& 3 represent shocked \ac{CBM} and \ac{GRB}
            %                jet respectively. They move together with the same \ac{LF} ($\gamma_2$, as viewed
            %                by a stationary observer in the unshocked \ac{CBM}), and have the same pressure but
            %                different densities.
            %                (Adapted from \citet{Kumar:2014upa}, figure~8)
            %            }
            %            \label{fig:aafg:theory:sr8}
            %            \end{figure*}
            %
            %            where subscripts $2$ and $1$ stand for downstream and upstream respectively, 
            %            $e'$ is the internal energy density, $n'$ is the proton number density, 
            %            $\gamma_{21}$ is the relative \ac{LF} of plasma in region 
            %            $2$ with respect to the region $1$
            %            $\gamma_{1s}$ is the relative \ac{LF} of plasma in region $1$ with respect to the shock front,
            %            $\hat{\gamma}$ is the adiabatic index of the fluid, for the ideal, relativistic 
            %            fluid is $\hat{\gamma}=4/3$ and subrelativisitc $\hat{\gamma}=5/3$.
            %            The $2$ and $1$ regions are also shown in Fig.~\ref{fig:aafg:theory:sr8} 
            %            (see also \cite{Nava:2013} for a more comprehensive take).
            %            %
            %            Solving the system Eq.~\eqref{eq:afterglow:blast} gives the full evolution 
            %            of the \blast{}. 
            
            \textbf{Electrons}: continuity eq. in energy space, 
            \begin{equation*}
            \frac{\partial }{\partial t}\frac{d n_e}{d\gamma_e} + \frac{\partial}{\partial \gamma_e}\Big[ \dot{\gamma_e}\frac{dn_e}{d\gamma_e} \Big] = S(\gamma_e), \rightarrow \frac{d n_e}{d\gamma_e} \propto \gamma^{-p}.
            \end{equation*}
            %
            Charactersitc, $\gamma_c$ , $\gamma_M$ , $\gamma_m$.
            %
            %            where $\dot{\gamma_e} = -\sigma_T B^2 \gamma_e^2 / (6\pi m_e c)$ is the rate at 
            %            which electron \ac{LF} changes due to losses, $S(\gamma_e)$ is the injection 
            %            rate of electrons into the system.
            %            %
            %            Assume that the minimum \ac{LF} of injected electrons is $\gamma_m$, \eg, 
            %            where $S(\gamma_e) = 0$ for $\gamma_e < \gamma_m$.
            %            %
            %            \begin{equation}
            %            dn_e/d\gamma_e \propto 
            %            \begin{cases}
            %            \gamma_e^{-2} &\text{ if } \gamma_c < \gamma_e < \gamma_m, \\
            %            \gamma_e^{-p-1} &\text{ if } \gamma_e > \gamma_c > \gamma_m
            %            \end{cases}
            %            \label{eq:afterglow:elec_dist}
            %            \end{equation}
            
    }};
    
    \node (t2) [anchor=center,scale=1,opacity=1] at ([shift={(3.9cm,-0.7cm)}]current page.center){
        \parbox{0.53\textwidth}{
            \textbf{Synchrotron emission}: %\citep{RybickiLightman:1985} --
            %
            %        \begin{equation*}
            %        P_{syn} = \frac{2q^4 E^2}{3c^3 m_e^2}
            %        \,
            %        \omega_{syn}\sim\frac{qB\gamma_e^2}{m_e c}
            %        %\,
            %        %P_{syn}(\nu_{syn}) \sim P_{syn}/\nu_{syn}
            %        \end{equation*}
            %
            \begin{equation*}
            f_{\nu} = \int_{\gamma_{\nu}}^{\infty} d\gamma_e \frac{dn_e}{d\gamma_e}P_{syn}(\nu) 
            \end{equation*}
            
            For the $dn_e/d\gamma_e\propto\gamma_e^{-p}$, the 
            $f_{\nu} = f_{\nu}(\nu_m,\nu_c)$
            
            \vspace{5mm}
            
            \textbf{Equal time arrival surface} integration
            %
            \begin{equation*}
            F(\nu,t) \sim c\int_0^{\theta}\int_{0}^{r}\frac{P'(\nu',t_{em},r)}{\Gamma^2(1-\beta\cos\theta)^2}r^2 dr d\cos\theta,
            \end{equation*}
            %with $c = (1+z) / (2d_L^2)$
            including relativistic effects.
    }};
}
\end{tikzpicture}
\end{frame}

%% =======================================================
%%
%%                   Method :: Code
%%
%% =======================================================

\begin{frame}{Method :: \texttt{PyBlastAfterglow}} %% ---------- title 

\begin{tikzpicture}[overlay,remember picture]

\uncover<1->{ % <-> |
\node (t1) [anchor=center,scale=1,opacity=1] at ([shift={(-3.9cm,-0.7cm)}]current page.center){
    \parbox{0.53\textwidth}{
        \textbf{Blast wave dynamics}
        \begin{itemize}
        \item[\textcolor{black}{$\blacksquare$}] uniform ISM; FS only$^\text{\citep{Peer:2012}}$
        \item[\textcolor{blue}{$\blacksquare$}] $\forall$ ISM; FS/RS; Rad.Loss$^\text{\citep{Nava:2013}}$
        \end{itemize}
        \textbf{Transrelativisitc equation of state}
        \begin{itemize}
        \item[\textcolor{black}{$\blacksquare$}]Assymptic$^\text{\citep{Nava:2013}}$
        \item[\textcolor{blue}{$\blacksquare$}] HD-simulation informed$^\text{\citep{Peer:2012}}$
        \end{itemize}
        \textbf{Lateral Spreading}
        \begin{itemize}
        \item[\textcolor{black}{$\blacksquare$}] Speed of sound limited$^\text{\citep{Huang:1999di}}$
        \item[\textcolor{blue}{$\blacksquare$}] 2D HD -informed$^\text{\citep{Granot:2012}}$ 
        \end{itemize}
}};
}

\uncover<1->{ % <-> |
\node (t2) [anchor=center,scale=1,opacity=1] at ([shift={(3.9cm,-0.7cm)}]current page.center){
    \parbox{0.53\textwidth}{
        \textbf{Electron distribution} 
        \begin{itemize}
        \item[\textcolor{black}{$\blacksquare$}] Simple 2-segment \ac{BPL}
        \item[\textcolor{blue}{$\blacksquare$}] \ac{BPL} with accurate crit. Lorentz fact.
        \item[\textcolor{gray}{$\blacksquare$}] $\forall$ segmented electron distribution
        \end{itemize}
        \textbf{Synchrotron spectrum}
        \begin{itemize}
        \item[\textcolor{black}{$\blacksquare$}] 3-segment \ac{BPL} $^\text{\citep{Sari:1997qe}}$
        \item[\textcolor{blue}{$\blacksquare$}] Smooth \ac{BPL}$^\text{\citep{Johannesson:2006zs}}$ + SSA$^\text{\citep{Johannesson:2018cdl}}$
        \item[\textcolor{gray}{$\blacksquare$}] Direct integration of syntronton function$^\text{\citep{Dermer:2009}}$
        \end{itemize}
        \textbf{Equal time arrival surface integrator}
        \begin{itemize}
        \item[\textcolor{black}{$\blacksquare$}] Piece-wise sum (including $\mathcal{D}$)$^\text{\citep{Fernandez:2021xce}}$
        \end{itemize}
}};
}
\end{tikzpicture}
\end{frame}


















%% ----------------------------------------------------------------
%%
%% Spiral Arms
%%
%% ----------------------------------------------------------------

%% FROM LETTER 

%From the fluid's stress energy tensor,
%we compute the angular momentum density flux $J_r = T_{ra}(\partial_\phi)^a$,
%where $\phi$ is the cylindrical angular coordinate;
%angular momentum is conserved if $(\partial_\phi)^a$ is a Killing vector.
%
%% FROM PAPER 

%The fluid's angular momentum analysis in the remnant and disk is performed
%assuming axisymmetry,
%%(see Appendix~\ref{app:ang} for derivation)
%that is, we assume $\phi^{\mu} = (\partial_{\phi})^{\mu}$ is a Killing
%vector. Accordingly, the conservation law (Eq.~\eqref{eq:theory:tmunu_eq_0}) 
%reads
%%
%\begin{equation}
%\partial_t(T^{\mu\nu}\phi_{\nu}n_{\nu}\sqrt{\gamma}) -
%\partial_i(\alpha T^{i \nu}\phi_{\nu}\sqrt{\gamma}) = 0 \ ,
%\end{equation}
%%
%where $n^\mu$ is the normal vector to the spacelike hypersurfaces of
%the spacetime's $3+1$ decomposition.
%%
%The equation implies the conservation of the angular momentum 
%\begin{equation}
%J = % \int j dV = 
%%- \int \, T_{\mu\nu}n^{\mu}\phi^{\nu}\,\dd ^3x = 
%-\int \,
%T_{\mu\nu}n^{\mu}\phi^{\nu}\,\sqrt{\gamma}\, \dd^3 x\ .
%\end{equation}
%%
%In the cylindrical coordinates $x^i=(r,\phi,z)$ adapted to the symmetry
%the angular momentum density is  
%%
%\begin{equation}
%j = %-
%\rho h W^2 v_{\phi} \ ,
%\label{eq:method:ang_mom}
%\end{equation}
%%
%and the angular momentum flux is 
%%
%\begin{equation}
%\alpha\sqrt{\gamma}T^r _{\nu}\phi^{\nu} =
%\alpha\sqrt{\gamma}\rho h W^2 (v^{r}v_{\phi}) .
%\end{equation}
%
%We evaluate these quantities from the 3D snapshots of our simulations.


\begin{frame}{Spiral arms in \pmerg{} remnant$^\text{\citep{Nedora:2019jhl}}$} %% ---------- title 

\begin{tikzpicture}[overlay,remember picture]

\uncover<1->{ % <-> |
    \node (t1) [anchor=center,scale=1,opacity=1] at ([shift={(-3.2cm,1.2cm)}]current page.center){
        \parbox{0.65\textwidth}{
            The angular momentum flux in the cylindrical coordinates.% $x^i=(r,\phi,z)$.
            % $\phi^{\nu}$ is the generator of the rotations in the orbital plane.
            \begin{equation*}
            \alpha\sqrt{\gamma}T^r _{\nu}\phi^{\nu} =
            \alpha\sqrt{\gamma}\rho h W^2 (v^{r}v_{\phi})\, ,
            \end{equation*}
            assiming $\phi^{\mu} = (\partial_{\phi})^{\mu}$ is a Killing vector. % angular momentum is conserved
    }};
}

\uncover<1-1>{ % <-> |
    \node (img1) [anchor=center,scale=1,opacity=1] at ([shift={(5.3cm,-0.6cm)}]current page.center){
        \parbox{0.5\textwidth}{
            \includegraphics[height=6cm]{raycasting_smooth_cropped.pdf}
            
            \hspace*{2.8mm}Spiral arms within the disk
    }};
}

%shocks are generated at the collisional interface of the \acp{NS} cores,
%as well as, tidal torques exerted by the non axisymmetric remnant result in a formation
%of the disk.
%Matter, energy and angular momentum are injected into the disk as the spiral 
%density waves propagate outwards from the mass-shedding \ac{MNS} remnant
\uncover<1->{ % <-> |
    \node (t2) [anchor=center,scale=1,opacity=1] at ([shift={(-3.5cm,-1.8cm)}]current page.center){
        \parbox{0.6\textwidth}{
            \begin{itemize}
                \item Shocks and tidal torques exerted from non-axisymmetric remnant $\rightarrow$ disk;
                \item spiral arms is a generic hydrodynamic effect.
            \end{itemize}
            Injection of matter, energy and angular momentum.
    }};
}

\end{tikzpicture}

\end{frame}


%% ----------------------------------------------------------------
%%
%% Angular momentum trasport
%%
%% ----------------------------------------------------------------

\begin{frame}{Angular momentum transport in \pmerg{} remnant$^\text{\citep{Nedora:2020pak}}$} %% ---------- title 

\begin{tikzpicture}[overlay,remember picture]

\uncover<1->{ % <-> |
    \node (img2) [anchor=center,scale=1,opacity=1] at ([shift={(4.8cm,-0.2cm)}]current page.center){
        \parbox{0.5\textwidth}{
            \includegraphics[height=5.5cm]{remnant/evol_jflux_2d_DD2_M13641364_M0_SR_R1.pdf}
            
            \hspace*{2.8mm}Ang. mom. transport through disk
    }};
}
%shocks are generated at the collisional interface of the \acp{NS} cores,
%as well as, tidal torques exerted by the non axisymmetric remnant result in a formation
%of the disk.
%Matter, energy and angular momentum are injected into the disk as the spiral 
%density waves propagate outwards from the mass-shedding \ac{MNS} remnant

\uncover<1->{ % <-> |
    \node (t2) [anchor=center,scale=1,opacity=1] at ([shift={(-3.8cm,-0.8cm)}]current page.center){
        \parbox{0.6\textwidth}{
            \begin{itemize}
                \item The $J$ is transported via spiral waves induced by $m=\{1,2\}$ modes.
                \item Spiral density modes inject ${\sim}0.1-0.4\,\Msun$ into the disk
                \item ${\sim}50\%$ of $J$ remnant $\rightarrow$ disk
                \item Mass injection is stronger in models with stiffer EOS. 
                \item Accretion, mass-shedding, weak processes
                \item Larger temperatures lead to lower rotational frequency at which the mass 
                shedding occurs% \citep{Kaplan:2013wra}. 
            \end{itemize}
    }};
}

\end{tikzpicture}

\end{frame}



%% ----------------------------------------------------------------
%%
%% Long-term Evolution
%%
%% ----------------------------------------------------------------



%\begin{frame}{Long term evolution \& remnant's fate$^\text{\citep{Nedora:2020pak}}$} %% ---------- title 
%
%%As the disk expands and cools, the recombination of nucleons into alpha particles 
%%starts to take place. The energy, released in recombination, might be sufficient 
%%for the outermost layers to become unbound, generating an outflow and contributing 
%%to the disk depletion \citep{Beloborodov:2008nx,Lee:2009uc,Fernandez:2013tya}.
%%%
%%This process however is expected to take place on timescales, longer than
%%those that are simulated here. On the $\sim100$~ms timescale, however, the outflows 
%%are driven by the neutrino heating, above the remnant, the so-called neutrino-driven 
%%wind (\nwind; \citep{Dessart:2008zd,Perego:2014fma,Just:2014fka}), and by the dynamical 
%%interactions between the \ac{MNS} remnant and the disk, the \swind{} \citep{Nedora:2019jhl}.
%%%
%%We discuss the properties of the \nwind{} found in our simulations
%%in Sec.~\ref{sec:bns_sims:nwind} and the properties of the 
%%\swind{} in Sec.~\ref{sec:bns_sims:sww}.
%
%\begin{tikzpicture}[overlay,remember picture]
%
%%We evaluate the amount of angular momentum lost to \acp{GW} following the 
%%\citet{Damour:2011fu,Bernuzzi:2012ci,Bernuzzi:2015rla}\footnote{
%%    The radiated angular momentum is computed from the 
%%    multipole loments $N_{lm}$ for the \ac{NR} complex "news function" at infinity. 
%%    The $J_{z;\text{rad}}(t)$ is computed as \citep{Damour:2011fu} 
%%    \begin{equation*}
%%    \Delta J_{z\text{rad}}(t) \frac{1}{16}\sum_{l,m}^{l_{max}}\int_{t_0}^{t} dt' m \mathcal{L}[h_{lm}(t')(N_{lm}(t'))^*],
%%    \end{equation*}
%%    where $h_{lm}$ is the multipolar metric waveform, 
%%    $N_{lm}(t) = dh_{lm}(t) / dt$, the news function, and $l_{max}=8$.
%%    The $J$ loss is metric dependent ($h$).
%%    The $h$ (strain???) is computed from $\Psi_4(t) = dN/dt = d^2h/dt^2$ by a 
%%    frequency-domain integration procedure with a low-frequency cut 
%%    $\omega_0 = 0.032/(m_1+m_2)$.
%%    The routines used for the calculation are taken from the scientific library
%%    \texttt{scidata}, available at \url{https://bitbucket.org/dradice/scidata}.
%%}.
%
%\uncover<1->{ % <-> |
%    \node (t1) [anchor=center,scale=1,opacity=1] at ([shift={(-3.1cm,0.5cm)}]current page.center){
%        \parbox{0.7\textwidth}{
%            \begin{itemize}
%                \item Remnant borm with excess in $M_b$ and $J$ (HMNS)
%                \item GWs remove some $J$
%                \item Massive outflows from expanded disk
%                driven by $m=\{1,2\}$ and $J$ transport
%                \item \swind{} Removes $J$ and $M_b$
%                \item The remannt can reach stable configuration
%                \item Study of remnant's fate requries long 3D NR simulations
%            \end{itemize}
%    }};
%}
%
%\uncover<1->{ % <-> |
%    \node (img1) [anchor=center,scale=1,opacity=1] at ([shift={(4.6cm,-0.2cm)}]current page.center){
%        \parbox{0.5\textwidth}{
%            \includegraphics[height=6cm]{ejecta_sec/secular_j_mb_RNS_blh.pdf}
%%            Baryon mass vs angular momentum diagram for the BLh $q=1$ remnant.
%%            The colored diamond marks the baryonic mass and angular momentum at the end
%%            of the dynamical gravitational-wave dominated phase.
%%            After the GW phase, the evolution is driven by the massive outflows.
%%            The solid black line is the $M_b$ and $J$ estimated from the 3D data
%%            integrals under the assumption of axisymmetry.
%%            The green dashed line is a conservative estimate
%%            of the mass ejection and a possible trajectory for the viscous
%%            evolution as estimated in \citet{Radice:2018xqa}. The crosses are
%%            a linear extrapolation in time of the solid black line. The gray
%%            shaded region is the region of stability of rigidly rotating NS equilibria.
%%            Adopted from \cite{Nedora:2020pak}
%    }};
%}
%
%%\uncover<1->{ % <-> |
%%    \node (t2) [anchor=center,scale=1,opacity=1] at ([shift={(-3.8cm,-1.8cm)}]current page.center){
%%        \parbox{0.6\textwidth}{
%%            \begin{itemize}
%%            \item The long-term evolution of the disk is driven by its interaction with the \ac{MNS} 
%%            remnant and cooling.
%%            \item Accretion vs. mass shedding
%%            \item spiral density waves inject mass and energy into disk and it heats up \& expands
%%            \item Softness \ac{EOS} (strong desntiy modes) \& high temps (allowed by \ac{EOS}) $\rightarrow$ mass-shedding
%%            \end{itemize}
%%    }};
%%}
%
%\end{tikzpicture}
%
%\end{frame}

%% --------------------------------------------------------------------------------------------
%%
%%
%%
%% --------------------------------------------------------------------------------------------

\begin{frame}{Disk mass evolution} %% ---------- title 

%During the merger, shocks, generated at the collisional interface of the \ac{NS} cores,
%as well as, tidal torques exerted by the non axisymmetric remnant, result in a formation
%of the disk. Matter, energy and angular momentum are injected into the disk as the spiral 
%density waves propagate outwards from the mass-shedding \ac{MNS} remnant 
%\citep{Bernuzzi:2015opx,Radice:2018xqa}.

\begin{tikzpicture}[overlay,remember picture]

\uncover<1->{ % <-> |
    \node (t1) [anchor=center,scale=1,opacity=1] at ([shift={(-3.8cm,1.0cm)}]current page.center){
        \parbox{0.6\textwidth}{
            The disk baryonic mass is computed as the volume integral of the conserved rest-mass density 
            %
            \begin{equation*}
            \label{eq:method:mdisk}
            M_{\text{disk}} = \int_{\rho>10^{13}\,\text{\gcm}} \sqrt{\gamma}~W\rho \dd^3 x
            \end{equation*}
    }};
}

\uncover<1->{ % <-> |
    \node (img1) [anchor=center,scale=1,opacity=1] at ([shift={(4.8cm,-0.2cm)}]current page.center){
        \parbox{0.5\textwidth}{
            \includegraphics[height=6cm]{disk/total_disk_mass_evo.pdf}
    }};
}

\uncover<1->{ % <-> |
    \node (t2) [anchor=center,scale=1,opacity=1] at ([shift={(-3.8cm,-2.2cm)}]current page.center){
        \parbox{0.6\textwidth}{
            \begin{itemize}
            \item The long-term evolution of the disk is driven by its interaction with the remnant and cooling.
            \item Accretion vs. mass shedding
            \item spiral density waves inject mass and energy into disk and it heats up \& expands
            \item Soft EOS (strong desntiy modes) \& high temps (allowed by EOS) $\rightarrow$ mass-shedding
            \end{itemize}
    }};
}

\end{tikzpicture}

\end{frame}

%% --------------------------------------------------------------------------------------------
%%
%%
%%
%% --------------------------------------------------------------------------------------------

\begin{frame}{The evolution of the mass-weighted electron fraction} %% ---------- title 

\begin{tikzpicture}[overlay,remember picture]

\uncover<1->{ % <-> |
    \node (t1) [anchor=center,scale=1,opacity=1] at ([shift={(-0.8cm,1.6cm)}]current page.center){
        \parbox{1.0\textwidth}{
            \begin{itemize}
            \item During the formation, shocks and spiral waves raise the disk $Y_e\sim0.25$
            \item The bulk, shielded from neutrinos, returns to $Y_e\lesssim0.1$.
            \item The outer part of the disk, is subjected to the strong irradiation and reaches $Y_e\sim0.4$.
            \end{itemize}
    }};
}

\uncover<1->{ % <-> |
    \node (img1) [anchor=center,scale=1,opacity=1] at ([shift={(5.5cm,-1.5cm)}]current page.center){
        \parbox{0.5\textwidth}{
            \includegraphics[height=5cm]{disk/final_disk_timecorr_blh_q1_Lk.pdf}
    }};
}

\uncover<1->{ % <-> |
    \node (img1) [anchor=center,scale=1,opacity=1] at ([shift={(-0.5cm,-1.5cm)}]current page.center){
        \parbox{0.5\textwidth}{
            \includegraphics[height=5cm]{disk/final_disk_timecorr_ls220_q14_LK.pdf}
    }};
}

\end{tikzpicture}

\end{frame}

%\includegraphics[width=0.49\textwidth]{disk/final_disk_timecorr_blh_q1_Lk.pdf}
%\includegraphics[width=0.49\textwidth]{disk/final_disk_timecorr_ls220_q14_LK.pdf}

%% --------------------------------------------------------------------------------------------
%%
%%
%%
%% --------------------------------------------------------------------------------------------


\begin{frame}{Final disk structure} %% ---------- title 

\begin{tikzpicture}[overlay,remember picture]

\uncover<1->{ % <-> |
    \node (t1) [anchor=center,scale=1,opacity=1] at ([shift={(-0.8cm,1.8cm)}]current page.center){
        \parbox{1.0\textwidth}{
            \begin{itemize}
            \item Low $Y_e$ inner disk
            \item High $Y_e$ outer disk
            \end{itemize}
    }};
}

\uncover<1->{ % <-> |
    \node (img1) [anchor=center,scale=1,opacity=1] at ([shift={(4.0cm,-1.5cm)}]current page.center){
        \parbox{0.5\textwidth}{
            \includegraphics[height=5cm]{disk/final_structure/slice_xz_entr_ye_blh_q1_rl3.pdf}
    }};
}

\uncover<1->{ % <-> |
    \node (img1) [anchor=center,scale=1,opacity=1] at ([shift={(-4.0cm,-1.5cm)}]current page.center){
        \parbox{0.5\textwidth}{
            \includegraphics[height=5cm]{disk/final_structure/slice_xy_entr_ye_blh_q1_rl3.pdf}
    }};
}

\end{tikzpicture}

\end{frame}

%\includegraphics[width=0.49\textwidth]{disk/final_structure/slice_xz_entr_ye_blh_q1_rl3.pdf}
%\includegraphics[width=0.49\textwidth]{disk/final_structure/slice_xy_entr_ye_blh_q1_rl3.pdf}

%% --------------------------------------------------------------------------------------------
%%
%%
%%
%% --------------------------------------------------------------------------------------------


\begin{frame}{Final disk properties \& composition} %% ---------- title 

\begin{tikzpicture}[overlay,remember picture]

\uncover<1->{ % <-> |
    \node (t1) [anchor=center,scale=1,opacity=1] at ([shift={(-0.8cm,1.8cm)}]current page.center){
        \parbox{1.0\textwidth}{
            \begin{itemize}
            \item Bimodal structure in $\langle Y_e \rangle$ and $\langle s \rangle$, second peak \ac{EOS} \& \mr{} dependent
            \item High $Y_e$ outer disk
            \end{itemize}
    }};
}

\uncover<1->{ % <-> |
    \node (img1) [anchor=center,scale=1,opacity=1] at ([shift={(0.0cm,-1.5cm)}]current page.center){
        \parbox{1.0\textwidth}{
            \includegraphics[height=5cm]{disk/final_structure/disk_hist_shared.pdf}
    }};
}

%\uncover<1->{ % <-> |
%    \node (img1) [anchor=center,scale=1,opacity=1] at ([shift={(-4.0cm,-1.5cm)}]current page.center){
%        \parbox{0.5\textwidth}{
%            \includegraphics[height=5cm]{disk/final_structure/slice_xy_entr_ye_blh_q1_rl3.pdf}
%    }};
%}

\end{tikzpicture}

\end{frame}



% ---------------------------------------------------------
%
% Spiral wave wind
%
% --------------------------------------------------------


\begin{frame}{Spiral-wave wind -- Mass}
\begin{tikzpicture}[overlay,remember picture]
\uncover<1->{ % <-> |
    \node (t1) [anchor=center,scale=1,opacity=1] at ([shift={(-3.1cm,0.5cm)}]current page.center){
        \parbox{0.7\textwidth}{
            Bernoulli criterion ($hu_t < -1$, stationary flow)\\
            \begin{itemize}
            \item is an idicator of the remnant lifetime,
            \item is larger if \ac{EOS} is softer (as $m=1$ is stronger) % stiffer \ac{EOS} leads to a more massive disk 
            \item is larger if $q>1$
            \item increases if turbulent viscocity is present
            \end{itemize}
            $M_{\text{wind}}$ is larger for more extended disks. % that in turn depend on thermal pressure
    }};
}
\uncover<1->{ % <-> |
    \node (img1) [anchor=center,scale=1,opacity=1] at ([shift={(4.2cm,-0.2cm)}]current page.center){
        \parbox{0.5\textwidth}{
            \includegraphics[height=6cm]{ejecta_postdyn/wind_mass_flux.pdf}
    }};
}
\end{tikzpicture}
\end{frame}

\begin{frame}{Spiral-wave wind \& neutrino-driven wind} %% ---------- title 

%% --------------------------------------------------------------------------------------------
%%
%%
%%
%% --------------------------------------------------------------------------------------------

%    \includegraphics[width=0.49\textwidth]{slices/slice_xz_ye_hu_1.pdf}
%\includegraphics[width=0.49\textwidth]{slices/slice_xz_abs_energy_hu_3.pdf}
%\includegraphics[width=0.49\textwidth]{slices/slice_xy_ye_hu_1.pdf}
%\includegraphics[width=0.49\textwidth]{slices/slice_xy_abs_energy_hu_3.pdf}

\begin{tikzpicture}[overlay,remember picture]

\uncover<1->{ % <-> |
    \node (t1) [anchor=center,scale=1,opacity=1] at ([shift={(-0.8cm,1.8cm)}]current page.center){
        \parbox{1.0\textwidth}{
            \begin{itemize}
            \item absorption of $\nu_e$ neutrinos raises the $Y_e$ of the polar outflow, drives outflow
            \item \nwind{} is not a steady state outflow, ${\sim}10^{-3}-10^{-4}M_{\odot}$
            %% The strongest neutrino heating occurs in the vicinity of the remnant at densities $\rho\sim10^{11}$~\gcm,
            \end{itemize}
    }};
}

\uncover<1->{ % <-> |
    \node (img1) [anchor=center,scale=1,opacity=1] at ([shift={(4.0cm,-1.5cm)}]current page.center){
        \parbox{0.5\textwidth}{
            \includegraphics[height=5cm]{slices/slice_xz_ye_hu_1.pdf}
    }};
}
%\uncover<2->{ % <-> |
%    \node (img1) [anchor=center,scale=1,opacity=1] at ([shift={(4.0cm,-1.5cm)}]current page.center){
%        \parbox{0.5\textwidth}{
%            \includegraphics[height=5cm]{slices/slice_xy_ye_hu_1.pdf}
%    }};
%}

%%% Right: $-hu_0$ and the absorption energy rate $Q_{\text{abs};\:\bar{\nu}_e}$ 
%%% of electron antineutrinos normalized to the fluid density $D$.
%%%  heating energy rate due to electron anti-neutrino absorption $Q_{\text{abs};\:\bar{\nu}_e}$
\uncover<1->{ % <-> |
    \node (img1) [anchor=center,scale=1,opacity=1] at ([shift={(-4.0cm,-1.5cm)}]current page.center){
        \parbox{0.5\textwidth}{
            \includegraphics[height=5cm]{slices/slice_xz_abs_energy_hu_3.pdf}
    }};
}
%\uncover<2->{ % <-> |
%    \node (img1) [anchor=center,scale=1,opacity=1] at ([shift={(-4.0cm,-1.5cm)}]current page.center){
%        \parbox{0.5\textwidth}{
%            \includegraphics[height=5cm]{slices/slice_xy_abs_energy_hu_3.pdf}
%    }};
%}

\end{tikzpicture}

\end{frame}