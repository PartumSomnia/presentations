%% --------------------------------------------------------------
%%
%% N R  S I M U L A T I O N S
%%
%% --------------------------------------------------------------
\begin{frame}{Method} %% ---------- title 
\begin{tikzpicture}[overlay,remember picture]
\uncover<1->{ % <-> |
    \node (t1) [anchor=center,scale=1,opacity=1] at ([shift={(-0.0cm,-0.2cm)}]current page.center){
        \parbox{1.1\textwidth}{
            A prime tool to understand these processes is NR models \\
            \wisky{} code$^\text{\citep{Radice:2013apa,Radice:2012cu,Radice:2013xpa,Radice:2013hxh,
                                 Radice:2015nva,Radice:2016dwd,Radice:2018pdn,Radice:2020ids}}$
            \begin{itemize}
                \item Eqs. of GR in Z4c formulation via FD method
                \item Eqs. of GRHD in a conservative form via KT central scheme
                \item Neutrino radiation transport via Leakage + M0 scheme
                \item Turbulent viscosity of magnetic origin via GRLES appraoch
                \item Microphsyical EOS with finite temperature effects
            \end{itemize}
            Overall, $37$ unique simulations targeted to \GW{}, (total of $76$, some $100+$~ms) \\
    }};
}
%
\uncover<2->{ % <-> |
    \node (t1) [anchor=center,scale=1,opacity=1] at ([shift={(-0.0cm,-3.4cm)}]current page.center){
        \parbox{1.1\textwidth}{
            \textbf{Goal}: Analyzed ejecta/\nuc{}/EM counterparts \& statistical analysis \\
            \textbf{Methods}: Post-processing pipeline \texttt{bns\_ppr\_tools} for ejecta/disk/remnant
%            \begin{itemize}
%            \item Developped post-processing pipeline \texttt{bns\_ppr\_tools}
%            \item \textbf{Goal}: Analyzed for ejecta/\nuc{}/EM counterparts \& statistical analysis
%%            \item Statisitcal analysis alongside other models in literature
%            \end{itemize}
    }};
}
\end{tikzpicture}
\end{frame}

%% ----------------------------------------------------------------------------

\begin{frame}{Hydrodynamic modes in the postmerger remnant$^\text{\citep{Nedora:2019jhl,Nedora:2020pak}}$} %% ---------- title 

\begin{tikzpicture}[overlay,remember picture]

\uncover<1->{ % <-> |
    \node (t1) [anchor=center,scale=1,opacity=1] at ([shift={(-3.6cm,1.5cm)}]current page.center){
        \parbox{0.6\textwidth}{
            Decomposition in Fourier modes $e^{-i m\phi}$ 
            of the Eulerian rest-mass density 
            \begin{equation*}
            \label{eq:modes}
            C_m(t) = \int \rho(x,y,z=0,t) e^{-i m \phi(x,y)} \text{d}x \text{d} y \, ,
            \end{equation*}
    }};
}
%
\uncover<1->{ % <-> |
    \node (img1) [anchor=center,scale=1,opacity=1] at ([shift={(4.5cm,-0.5cm)}]current page.center){
        \parbox{0.5\textwidth}{
            \includegraphics[height=6cm]{remnant/dens_modes/modes_rho_dd2.pdf}
            \hspace*{10mm} Density modes in the remnant 
            %Modes analysis for exemplary equal-mass long-live and short-lived
            %        remnants. The evolution of the $m=2$ and the $m=1$ monitored by
            %        Eq.~\eqref{eq:modes} is shown for the DD2 and LS220 remnant with and
            %        without turbulent viscosity. The $m=2$ mode in the long-lived
            %        remnant is strongly damped by the emission of gravitational
            %        radiation and becomes comparable to the $m=1$ mode on a timescale of
            %        ${\gtrsim}20\,$ms. Turbulent viscosity sustain the $m=2$ mode for
            %        a longer period. The $m=2$ mode is instead dominant to collapse in
            %        the short-lived remnant.
            %        (Adopted from \cite{Nedora:2020pak})
    }};
}

\uncover<1->{ % <-> |
    \node (t2) [anchor=center,scale=1,opacity=1] at ([shift={(-3.8cm,-1.8cm)}]current page.center){
        \parbox{0.6\textwidth}{
            \begin{itemize}
                \item Newly born remnant is not axisymmetric
                \item Bar-shaped $m=2$ dominats the \textbf{early} phase
                and later $m=2$ is damped by the emission of \acp{GW}
                \item One-armed $m=1$ is present on ${\sim}100$~ms 
                \item $m=1$ mode is stronger if \mr{} is higher
                \item And as the $m=1$ modes are not efficiently damped$^\text{\citep{Paschalidis:2015mla,Radice:2016gym,Lehner:2016wjg,East:2016zvv}}$
            \end{itemize}
    }};
}

\end{tikzpicture}





%\begin{itemize}
%    \item Not axisymmetric. Angular momentum redistribution, \acp{GW} losses %\cite{Bernuzzi:2015opx,Radice:2018xqa}
%    \item Long-term evolution driven by viscous processes \& weak interactions
%    \item After \ac{GW} losses \ac{MNS} have excess in $M$ with respect to the cold, rigid.rot equilibria %\citep{Radice:2018xqa}.
%    \item \ac{MNS} remnant evolves towards stability at the mass-shedding limit
%    \item Collapse to a \ac{BH} depends on post-merger state, temperature and composition
%\end{itemize}
%
%\begin{figure}[t]
%    \centering
%    \includegraphics[width=0.49\textwidth]{raycasting_smooth_cropped.pdf}
%    \caption{3D distribution of angular momentum density flux $J_r$
%        from the DD2 simulation with turbulent viscosity at ${\sim}43.5$~ms after
%        merger. $J_r$ is shown on a central region of
%        $(89\times89\times60)$~km${}^3$ covering the remnant NS
%        and disk, and it is given in units where $c=G=\Msun=1$.
%        (Adapted from \citet{Nedora:2019jhl})
%    }
%    \label{fig:ang_mom_flux}
%\end{figure}
%
%\begin{itemize}
%    \item High $q$ models undergo prompt collapse (no core bounce) % central density monotonically rises\citep{Radice:2020ddv,Bernuzzi:2020tgt,Bernuzzi:2020txg}. 
%    \item 
%\end{itemize}

\end{frame}

%% ----------------------------------------

%% FROM LETTER 

%From the fluid's stress energy tensor,
%we compute the angular momentum density flux $J_r = T_{ra}(\partial_\phi)^a$,
%where $\phi$ is the cylindrical angular coordinate;
%angular momentum is conserved if $(\partial_\phi)^a$ is a Killing vector.
%
%% FROM PAPER 

%The fluid's angular momentum analysis in the remnant and disk is performed
%assuming axisymmetry,
%%(see Appendix~\ref{app:ang} for derivation)
%that is, we assume $\phi^{\mu} = (\partial_{\phi})^{\mu}$ is a Killing
%vector. Accordingly, the conservation law (Eq.~\eqref{eq:theory:tmunu_eq_0}) 
%reads
%%
%\begin{equation}
%\partial_t(T^{\mu\nu}\phi_{\nu}n_{\nu}\sqrt{\gamma}) -
%\partial_i(\alpha T^{i \nu}\phi_{\nu}\sqrt{\gamma}) = 0 \ ,
%\end{equation}
%%
%where $n^\mu$ is the normal vector to the spacelike hypersurfaces of
%the spacetime's $3+1$ decomposition.
%%
%The equation implies the conservation of the angular momentum 
%\begin{equation}
%J = % \int j dV = 
%%- \int \, T_{\mu\nu}n^{\mu}\phi^{\nu}\,\dd ^3x = 
%-\int \,
%T_{\mu\nu}n^{\mu}\phi^{\nu}\,\sqrt{\gamma}\, \dd^3 x\ .
%\end{equation}
%%
%In the cylindrical coordinates $x^i=(r,\phi,z)$ adapted to the symmetry
%the angular momentum density is  
%%
%\begin{equation}
%j = %-
%\rho h W^2 v_{\phi} \ ,
%\label{eq:method:ang_mom}
%\end{equation}
%%
%and the angular momentum flux is 
%%
%\begin{equation}
%\alpha\sqrt{\gamma}T^r _{\nu}\phi^{\nu} =
%\alpha\sqrt{\gamma}\rho h W^2 (v^{r}v_{\phi}) .
%\end{equation}
%
%We evaluate these quantities from the 3D snapshots of our simulations.


%% ----------------------------------------------------------------
%%
%% Spiral Arms
%%
%% ----------------------------------------------------------------

\begin{frame}{Spiral arms in \pmerg{} remnant$^\text{\citep{Nedora:2019jhl}}$} %% ---------- title 

\begin{tikzpicture}[overlay,remember picture]

\uncover<1->{ % <-> |
    \node (t1) [anchor=center,scale=1,opacity=1] at ([shift={(-3.2cm,1.2cm)}]current page.center){
        \parbox{0.65\textwidth}{
            The angular momentum flux in the cylindrical coordinates.% $x^i=(r,\phi,z)$.
            % $\phi^{\nu}$ is the generator of the rotations in the orbital plane.
            \begin{equation*}
            \alpha\sqrt{\gamma}T^r _{\nu}\phi^{\nu} =
            \alpha\sqrt{\gamma}\rho h W^2 (v^{r}v_{\phi})\, ,
            \end{equation*}
            assiming $\phi^{\mu} = (\partial_{\phi})^{\mu}$ is a Killing vector. % angular momentum is conserved
    }};
}

\uncover<1-1>{ % <-> |
    \node (img1) [anchor=center,scale=1,opacity=1] at ([shift={(5.3cm,-0.6cm)}]current page.center){
        \parbox{0.5\textwidth}{
            \includegraphics[height=6cm]{raycasting_smooth_cropped.pdf}
            
            \hspace*{2.8mm}Spiral arms within the disk
    }};
}

%shocks are generated at the collisional interface of the \acp{NS} cores,
%as well as, tidal torques exerted by the non axisymmetric remnant result in a formation
%of the disk.
%Matter, energy and angular momentum are injected into the disk as the spiral 
%density waves propagate outwards from the mass-shedding \ac{MNS} remnant
\uncover<1->{ % <-> |
    \node (t2) [anchor=center,scale=1,opacity=1] at ([shift={(-3.5cm,-1.8cm)}]current page.center){
        \parbox{0.6\textwidth}{
            \begin{itemize}
                \item Shocks and tidal torques exerted from non-axisymmetric remnant $\rightarrow$ disk;
                \item spiral arms is a generic hydrodynamic effect.
            \end{itemize}
            Injection of matter, energy and angular momentum.
    }};
}

\end{tikzpicture}

\end{frame}


%% ----------------------------------------------------------------
%%
%% Angular momentum trasport
%%
%% ----------------------------------------------------------------

\begin{frame}{Angular momentum transport in \pmerg{} remnant$^\text{\citep{Nedora:2020pak}}$} %% ---------- title 

\begin{tikzpicture}[overlay,remember picture]

\uncover<1->{ % <-> |
    \node (img2) [anchor=center,scale=1,opacity=1] at ([shift={(4.8cm,-0.2cm)}]current page.center){
        \parbox{0.5\textwidth}{
            \includegraphics[height=5.5cm]{remnant/evol_jflux_2d_DD2_M13641364_M0_SR_R1.pdf}
            
            \hspace*{2.8mm}Ang. mom. transport through disk
    }};
}
%shocks are generated at the collisional interface of the \acp{NS} cores,
%as well as, tidal torques exerted by the non axisymmetric remnant result in a formation
%of the disk.
%Matter, energy and angular momentum are injected into the disk as the spiral 
%density waves propagate outwards from the mass-shedding \ac{MNS} remnant

\uncover<1->{ % <-> |
    \node (t2) [anchor=center,scale=1,opacity=1] at ([shift={(-3.8cm,-0.8cm)}]current page.center){
        \parbox{0.6\textwidth}{
            \begin{itemize}
                \item The $J$ is transported via spiral waves induced by $m=\{1,2\}$ modes.
                \item Spiral density modes inject ${\sim}0.1-0.4\,\Msun$ into the disk
                \item ${\sim}50\%$ of $J$ remnant $\rightarrow$ disk
                \item Mass injection is stronger in models with stiffer EOS. 
                \item Accretion, mass-shedding, weak processes
                \item Larger temperatures lead to lower rotational frequency at which the mass 
                shedding occurs% \citep{Kaplan:2013wra}. 
            \end{itemize}
    }};
}

\end{tikzpicture}

\end{frame}



%% ----------------------------------------------------------------
%%
%% Long-term Evolution
%%
%% ----------------------------------------------------------------



\begin{frame}{Long term evolution \& remnant's fate$^\text{\citep{Nedora:2020pak}}$} %% ---------- title 

%As the disk expands and cools, the recombination of nucleons into alpha particles 
%starts to take place. The energy, released in recombination, might be sufficient 
%for the outermost layers to become unbound, generating an outflow and contributing 
%to the disk depletion \citep{Beloborodov:2008nx,Lee:2009uc,Fernandez:2013tya}.
%%
%This process however is expected to take place on timescales, longer than
%those that are simulated here. On the $\sim100$~ms timescale, however, the outflows 
%are driven by the neutrino heating, above the remnant, the so-called neutrino-driven 
%wind (\nwind; \citep{Dessart:2008zd,Perego:2014fma,Just:2014fka}), and by the dynamical 
%interactions between the \ac{MNS} remnant and the disk, the \swind{} \citep{Nedora:2019jhl}.
%%
%We discuss the properties of the \nwind{} found in our simulations
%in Sec.~\ref{sec:bns_sims:nwind} and the properties of the 
%\swind{} in Sec.~\ref{sec:bns_sims:sww}.

\begin{tikzpicture}[overlay,remember picture]

%We evaluate the amount of angular momentum lost to \acp{GW} following the 
%\citet{Damour:2011fu,Bernuzzi:2012ci,Bernuzzi:2015rla}\footnote{
%    The radiated angular momentum is computed from the 
%    multipole loments $N_{lm}$ for the \ac{NR} complex "news function" at infinity. 
%    The $J_{z;\text{rad}}(t)$ is computed as \citep{Damour:2011fu} 
%    \begin{equation*}
%    \Delta J_{z\text{rad}}(t) \frac{1}{16}\sum_{l,m}^{l_{max}}\int_{t_0}^{t} dt' m \mathcal{L}[h_{lm}(t')(N_{lm}(t'))^*],
%    \end{equation*}
%    where $h_{lm}$ is the multipolar metric waveform, 
%    $N_{lm}(t) = dh_{lm}(t) / dt$, the news function, and $l_{max}=8$.
%    The $J$ loss is metric dependent ($h$).
%    The $h$ (strain???) is computed from $\Psi_4(t) = dN/dt = d^2h/dt^2$ by a 
%    frequency-domain integration procedure with a low-frequency cut 
%    $\omega_0 = 0.032/(m_1+m_2)$.
%    The routines used for the calculation are taken from the scientific library
%    \texttt{scidata}, available at \url{https://bitbucket.org/dradice/scidata}.
%}.

\uncover<1->{ % <-> |
    \node (t1) [anchor=center,scale=1,opacity=1] at ([shift={(-3.1cm,0.5cm)}]current page.center){
        \parbox{0.7\textwidth}{
            \begin{itemize}
                \item Remnant borm with excess in $M_b$ and $J$ (HMNS)
                \item GWs remove some $J$
                \item Massive outflows from expanded disk
                driven by $m=\{1,2\}$ and $J$ transport
                \item \swind{} Removes $J$ and $M_b$
                \item The remannt can reach stable configuration
                \item Study of remnant's fate requries long 3D NR simulations
            \end{itemize}
    }};
}

\uncover<1->{ % <-> |
    \node (img1) [anchor=center,scale=1,opacity=1] at ([shift={(4.6cm,-0.2cm)}]current page.center){
        \parbox{0.5\textwidth}{
            \includegraphics[height=6cm]{ejecta_sec/secular_j_mb_RNS_blh.pdf}
%            Baryon mass vs angular momentum diagram for the BLh $q=1$ remnant.
%            The colored diamond marks the baryonic mass and angular momentum at the end
%            of the dynamical gravitational-wave dominated phase.
%            After the GW phase, the evolution is driven by the massive outflows.
%            The solid black line is the $M_b$ and $J$ estimated from the 3D data
%            integrals under the assumption of axisymmetry.
%            The green dashed line is a conservative estimate
%            of the mass ejection and a possible trajectory for the viscous
%            evolution as estimated in \citet{Radice:2018xqa}. The crosses are
%            a linear extrapolation in time of the solid black line. The gray
%            shaded region is the region of stability of rigidly rotating NS equilibria.
%            Adopted from \cite{Nedora:2020pak}
    }};
}

%\uncover<1->{ % <-> |
%    \node (t2) [anchor=center,scale=1,opacity=1] at ([shift={(-3.8cm,-1.8cm)}]current page.center){
%        \parbox{0.6\textwidth}{
%            \begin{itemize}
%            \item The long-term evolution of the disk is driven by its interaction with the \ac{MNS} 
%            remnant and cooling.
%            \item Accretion vs. mass shedding
%            \item spiral density waves inject mass and energy into disk and it heats up \& expands
%            \item Softness \ac{EOS} (strong desntiy modes) \& high temps (allowed by \ac{EOS}) $\rightarrow$ mass-shedding
%            \end{itemize}
%    }};
%}

\end{tikzpicture}

\end{frame}

%% ----------------------------------------------------------------
%%
%% Spiral wave wind
%%
%% ----------------------------------------------------------------

\begin{frame}{Spiral-wave wind -- Properties$^\text{\citep{Nedora:2020pak}}$}
\begin{tikzpicture}[overlay,remember picture]
\uncover<1->{ % <-> |
    \uncover<1->{ % <-> |
        \node (t1) [anchor=center,scale=1,opacity=1] at ([shift={(-2.9cm,1.6cm)}]current page.center){
            \parbox{0.7\textwidth}{
                Bernoulli criterion ($hu_t < -1$, stationary flow)
                \begin{itemize}
                    \item generic for long-lived remnants
                    \item larger if \ac{EOS} is softer or $q>1$
                    \item increases if turbulent viscocity is present
                \end{itemize}
%                $M_{\text{wind}}$ is larger for more extended disks. % that in turn depend on thermal pressure
        }};
    }
    \node (t1) [anchor=center,scale=1,opacity=1] at ([shift={(9.0cm,1.6cm)}]current page.center){
        \parbox{1.0\textwidth}{
            Properties of \swind:
            \begin{itemize}
                \item $\amw\propto t_{\text{coll}} \gg \amd$
                \item $0.1\lesssim \ayw\lesssim0.4$ peak ${\simeq}0.35$
                \item $\avw \sim 0.1\,$c / ${\sim}0.2\,$c
           %\item broad distribution around the binary plane, high $0.1\lesssim \ayw\lesssim0.4$ peak ${\simeq}0.35$.%$\langle Y_e \rangle$
            %\item The low electron fraction material originates primarily at early times, when the material did not have enough time to be processed by neutrinos and before the outflow reaches quasi-steady state.
            %\item $\avw$ is ${\sim}0.1\,$c / ${\sim}0.2\,$c for soft/still \acp{EOS}
            \end{itemize}
    }};
}
\uncover<1->{ % <-> |
    \node (img1) [anchor=center,scale=1,opacity=1] at ([shift={(1.5cm,-2.0cm)}]current page.center){
        \parbox{1.0\textwidth}{
            \includegraphics[height=4.5cm]{ejecta_postdyn/wind_hists_shared.pdf}
    }};
}
\end{tikzpicture}
\end{frame}